\documentclass[11pt, oneside, letter]{article}

\usepackage[margin=1in]{geometry}
\usepackage{titlesec}
\usepackage{titling}
\usepackage[numbers]{natbib}
\usepackage[american]{babel}
\usepackage{comment}
\usepackage[T1]{fontenc}
\usepackage{kpfonts}%  for math    
\usepackage{libertine}%  serif and sans serif
\usepackage[scaled=0.85]{beramono}%% mono
\usepackage[format=plain,indention=.3cm, font=footnotesize, labelfont=bf]{caption}
\usepackage{graphicx}
\usepackage[colorlinks,pdfstartview=FitH,linkcolor=black,citecolor=NavyBlue,urlcolor=NavyBlue,filecolor=black]{hyperref}
\usepackage{wrapfig}
\usepackage{fancyhdr,lastpage}
\pagestyle{fancy}
%\pagestyle{empty}      % Uncomment this to get rid of page numbers
\fancyhf{}\renewcommand{\headrulewidth}{0pt}
\rfoot{Page \thepage \hspace{1pt} of \pageref{LastPage}}

\usepackage{amsmath,amssymb,enumerate}
\usepackage{mdwlist}

\let\proof\relax 
\let\endproof\relax
\usepackage{amsthm}
\usepackage{amsfonts}
\usepackage{setspace}
\usepackage{booktabs}
\usepackage[usenames,dvipsnames,svgnames,table]{xcolor}
\usepackage{mathtools}
\usepackage{algorithm, algorithmicx, algpseudocode}
\usepackage{blindtext}
\usepackage{gensymb}
\usepackage{xparse}
\usepackage{mathrsfs}
\usepackage[mathscr]{euscript}
\usepackage[caption=false,font=footnotesize]{subfig}

\DeclareMathOperator*{\argmin}{arg\,min}
\DeclareMathOperator*{\argmax}{arg\,max}
\newcommand{\bigO}[1]{\Ocal(#1)} % Big O
\newcommand{\EV}[1]{\mathbb{E}[#1]} % Expected Value
\newcommand{\Var}[1]{\mathbb{V}[#1]} % Variance
\newcommand{\Cov}[1]{\mathbf{Cov}[#1]} % Covariance
\newcommand{\transpose}{\mathsf{T}}
\newcommand{\SO}{\mathrm{SO}}
\newcommand{\SE}{\mathrm{SE}}
\newcommand{\GL}{\mathrm{GL}}

\newcommand{\diag}{\mathop{\mathrm{diag}}}
\newcommand{\m}{\mathop{\mathrm{m}}}
\newcommand{\rad}{\mathop{\mathrm{rad}}}
\newcommand{\dBm}{\mathop{\mathrm{dBm}}}
\newcommand{\Hz}{\mathop{\mathrm{Hz}}}
\newcommand{\GHz}{\mathop{\mathrm{GHz}}}
\newcommand{\MHz}{\mathop{\mathrm{MHz}}}


\begin{document}

\title{\huge\textbf{NA 568 Mobile Robotics: Methods \& Algorithms Winter 2021 -- Homework 1}}
\author{AuthorFirst AuthorLast, umich uniqname}
\date{\today}
\maketitle


\section{Probability Basics (30 points)}
\begin{enumerate}[A.]
\item 

\item 

\item 

\item 
    
\item 
    
\end{enumerate}


\section{Bayes Filter (30 points)}

\begin{enumerate}[A.]

\item 

\item 

\item 

\end{enumerate}



\section{Bayes Filter (12 points)} 

Your answer ...

\section{Normal Random Variables (8 points)}

\begin{enumerate}
\item 
\item 
\item  
\item 
\end{enumerate}

\section{Cholesky Decomposition (8 points)}
\begin{enumerate}
\item[] Your answer ...

\item[] Your answer ...

\item[] Your answer ...

\item[] Your answer ...

\end{enumerate}


\section{Probability and Uncertainty Propagation (12 points)}

Your answer ...


\end{document}
